\documentclass[a4paper,10pt]{article}

\usepackage[utf8]{inputenc}
\usepackage[francais]{babel}
\usepackage[T1]{fontenc}
\usepackage{graphicx}
\usepackage[top=2cm, bottom=2cm, left=2cm, right=2cm]{geometry}
\usepackage{listings}
\lstset{
basicstyle=\small,
language=Prolog,
frame=single,
}


\usepackage[dvips]{hyperref}

\title{Rapport \\ Bases de Connaissances}
\author{Selove OKE CODJO \\ Solène EHOLIE}
\date{04/06/2014}

\begin{document}

 \maketitle
 \newpage
 \tableofcontents
 \newpage
 \section{Introduction}
 
 
 \section{Rendu des programmes r\'{e}alis\'{e}s en TP}
 
  \subsection{TP1 : generation de plan}
  
  Le but du TP1 était de prendre en main la programmation en prolog \`{a} travers un problème simple mettant en scène un agent et un objet. L'agent
  est capable de se d\'{e}placer d'un point a \`{a} un point b, de prendre l'objet et de le poser. Il va donc falloir g\'{e}n\'{e}rer le plan \`{a}
  suivre par l'agent pour aller d'une certaine situation \`{a} une autre
  
  
   \subsubsection{R\'{e}solution du probl\`{e}me}
  La premi\`{e}re chose \`{a} faire \'{e}tait de d\'{e}finir les diff\'{e}rentes actions dont notre agent est capable.
  Une action \'{e}tant d\'{e}finie par une specification, une liste de conditions n\'{e}cessaire \`{a} son accomplissement, une liste de propri\'{e}t\'{e}s qui ne seront
  plus vraies et une liste de nouvelles propri\'{e}t\'{e}s.
  
  Ainsi, on d\'{e}finit l'action aller demandant \`{a} un robot de se deplacer d'un point X \`{a} un point Y. il faut donc que le robot soit en X,
  propri\'{e}t\'{e} qui deviendra fausse apr\`{e}s le déplacement o\`{u} il se retrouve en Y. La d\'{e}finition est donn\'{e}e ci-dessous.
  \begin{lstlisting}
   action(aller(robot,X,Y), 
          [lieu(robot) = X], 
          [lieu(robot) = X], [lieu(robot) = Y]) :-
	      member(X,[a,b]), member(Y,[a,b]), X \= Y.
  \end{lstlisting}
  Ensuite on d\'{e}finit prendre, il faut que le robot et la boite soient au m\^{e}me endroit et que la main du robot soit libre, bien s\^{u}r
  la main n'est plus libre apr\`{e} l'action et l'objet n'est plus \`{a} l'endroit o\`{u} il \'{e}tait mais dans la main du robot.
  \begin{lstlisting}
   action(prendre(robot,O), 
          [lieu(robot) = L, lieu(O) = L, libre(main(robot))], 
          [libre(main(robot)), lieu(O)=L], 
          [lieu(O)=main(robot)]) :-
	      member(L,[a,b]), member(O,[boite]).
  \end{lstlisting}
  Enfin, on d\'{e}finit poser, l'objet doit \^{e}tre dans la main du robot avant de la quitter et se retrouver \`{a} la m\^{e}me position que le robot.
  \begin{lstlisting}
   action(poser(robot,O), 
          [lieu(robot) = L, lieu(O) = main(robot)], 
          [lieu(O)=main(robot)], 
          [lieu(O) = L, libre(main(robot))]) :- 
	      member(L, [a,b]), member(O, [boite]).
  \end{lstlisting}
  Il faut maintenant decrire ce que c'est qu'une transition entre un \'{e}tat E et un autre F, cele consiste juste en la réalisation d'une action dans
  E, il faut donc que les conditions soient v\'{e}rifi\'{e}es en E, il faut supprimer les propri\'{e}t\'{e}s qui ne seront plus v\'{e}rifi\'{e}es et 
  ajouter les nouvelles. On a donc :
  \begin{lstlisting}
   transition(A,E,F) :- action(A, C, S, AJ), verifcond(C,E), 
			 suppress(S, E, EI), ajouter(AJ,EI,F).
  \end{lstlisting}
  La fonction verifcond(C,E) v\'{e}rifi\'{e}e l'inclusion de C dans E, suppress(S, E, EI) supprime S de E pour donner EI et ajouter(AJ,EI,F)
  ajoute AJ \`{a} EI pour obtenir F.
  \begin{lstlisting}
   %verifcond(C,E) est l inclusion de C dans E
   verifcond([], _). %la liste vide est toujours incluse
   verifcond([C|LC], L) :- member(C,L), verifcond(LC,L).
  \end{lstlisting}
  \begin{lstlisting}
   %suppression
   suppress([],L,L).
   suppress([X|Y], Z, T) :- delete(Z,X,U), suppress(Y,U,T).
  \end{lstlisting}
  \begin{lstlisting}
   %ajout
   ajouter(AJ,E,F) :- union(AJ,E,F).
  \end{lstlisting}
  Pour finir, il nous faut generer un plan d'une certaine profondeur, et une autre qui nous permet d'entrer la profondeur que l'on veut sans avoir
  \`{a} modifier le code ces deux fonctions sont données ci-dessous :
  \begin{lstlisting}
   %generation de plan
   genere(E,F,[A],1):-
      transition(A,E,F).

   genere(EI,EF,[ACT|PLAN],M):-
      M > 1, transition(ACT,EI,E), N is M-1, 
      between(1,N,P), genere(E,EF,PLAN,P).
  \end{lstlisting}
  \begin{lstlisting}
   planifier(Plan) :-
      init(E),
      but(B),
      nl,
      write(' Profondeur limite : '),
      read(Prof),
      nl,
      genere(E,F,Plan,Prof),
      verifcond(B,F).
  \end{lstlisting}
   \subsubsection{Tests}
   
  \subsection{TP2 : planification multi-agents}
 L'objectif ici \'{e}tait de faire de la planification comme au TP1 à la seule différence qu'il y a ici plusieurs agents différents.
 Ce qui offre un nombre de situations plus \'{e}lev\'{e} et des cas plus complexes \`{a} traiter.
 
   \subsubsection{R\'{e}solution du probl\`{e}me}
   \subsubsection{Tests}
   
  \subsection{TP3 : g\'{e}n\'{e}rateur-d\'{e}monstrateur en logique modale}
  
   \subsubsection{R\'{e}solution du probl\`{e}me}
   \subsubsection{Tests}
   
  \subsection{TP4 : g\'{e}n\'{e}rateur de plan en logique modale}
  
   \subsubsection{R\'{e}solution du probl\`{e}me}
   \subsubsection{Tests}
   
   
 \section{Travaux r\'{e}alis\'{e}s lors du bureau d'\'{e}tude}
 
  \subsection{Etude de Lm\{p\} }
  
  \subsection{Etude de Lm\{E0\}}
  
  \subsection{Etude d'autres logiques}
  
 \section{Conclusion}
    
\end{document}